\documentclass[a4paper,11pt,brazil]{article}

\usepackage[left=25mm,top=15mm]{geometry}
\usepackage[utf8]{inputenc}
\usepackage[T1]{fontenc}
\usepackage{babel}
\usepackage{hyperref}
\usepackage{lmodern}
\usepackage{indentfirst}
\usepackage{tabularx}
\usepackage{graphics}
\usepackage{float}
\usepackage{newfloat}
\usepackage{syntax}
\usepackage{dirtree}
\usepackage{pygmentex}
\usepackage{tcolorbox}
\usepackage{enumitem}
\usepackage{tikz}
\usepackage{tikz-qtree}
\usepackage{mystyle}

\DeclareFloatingEnvironment[name=Gramática]{gramatica}

\setpygmented{boxing method=tcolorbox,lang=ocaml,tabsize=2}
\efboxsetup{hidealllines,backgroundcolor=yellow!30}

% syntax configuration
\renewcommand{\syntleft}{\normalfont\slshape\hspace{0.25em}}
\renewcommand{\syntright}{\hspace{0.25em}}
\renewcommand{\ulitleft}{\normalfont\ttfamily\bfseries\frenchspacing\color{red}\hspace{0.25em}}
\renewcommand{\ulitright}{\hspace{0.25em}}

\usetikzlibrary{graphs,graphdrawing}
\usegdlibrary{trees}

\tikzset{
  parser tree/.style={
    % level/.style={sibling distance=130mm/#1},
    % level distance=14mm,
    sibling distance=0.6em,
    % edge from parent fork down,
    every internal node/.style={draw,rounded corners},
    every leaf node/.style={fill=blue!10,rounded corners,very thin},
    edge from parent/.style={draw},
  },
} 


\renewcommand{\DTcomment}[1]{\hfill \sffamily #1}

\newcommand{\lang}{\textsl{Trivia}}


\begin{document}

\begin{center}
  \textbf{Construção de Compiladores I [BCC328]}
\end{center}
\begin{tcolorbox}[colback=yellow!20]
  \begin{center}
    \textbf{\Large Definição e implementação da linguagem \lang{}}\\[1em]
  \end{center}
\end{tcolorbox}
\begin{center}
  Departamento de Computação\par
  Universidade Federal de Ouro Preto\par
  Prof. José Romildo Malaquias\par
  \today
\end{center}

\begin{abstract}
  \lang{} é uma pequena linguagem de programação usada para fins
  didáticos na aprendizagem de técnicas de construção de
  compiladores.
  
  Na disciplina de construção de compiladores serão propostas
  atividades de documentação e implementação de \lang{}.
\end{abstract}

\tableofcontents

\section{A linguagem \lang{}}

No livro
\href{https://www.springer.com/gp/book/9783319669656}{Introduction to
  Compiler Design}, Torben apresenta uma pequena linguagem de
programação para fins didáticos. Nesta disciplina vamos considerar uma
linguagem semelhante, que chamaremos de \lang{}, baseada na linguagem
apresentada no livro, porém com algumas diferenças sintáticas e
semânticas.
\begin{center}
  \includegraphics[width=.4\textwidth]{trivia.png}
\end{center}

\lang{} é uma linguagem de programação bastante simples que será usada
para a prática de técnicas usadas na implementação de compiladores.

As construções da linguagem serão detalhadas no decorrer do
curso. Começaremos com uma versão básica, e oportunamente serão
apresentadas versões mais aprimoradas, com novas construções.

\lang{} é uma \textbf{linguagem funcional de primeira ordem} com
definições recursivas e tipagem estática. A sintaxe é apresentada na
gramática~\ref{grm:trivial}.

\begin{gramatica}
  \begin{synshorts}
    \begin{mdframed}
    \begin{tabular}{r@{$\;\rightarrow\;$}l>{\bfseries}r}
      <Program> & <Funs>                               & programa                \\[.9em]
      <Funs>    & <Fun>                                &                         \\
      <Funs>    & <Fun> <Funs>                         &                         \\[.9em]
      <Fun>     & <TypeId> "(" <TypeIds> ")" "=" <Exp> & declaração de função    \\[.9em]
      <TypeId>  & "int" "id"                           & tipo e identificador    \\
      <TypeId>  & "bool" "id"                          &                         \\[.9em]
      <TypeIds> & <TypeId>                             & lista de parâmetros     \\
      <TypeIds> & <TypeId> "," <TypeIds>               &                         \\[.9em]
      <Exp>     & "num"                                & literal inteiro         \\
      <Exp>     & "id"                                 & variável                \\
      <Exp>     & <Exp> "+" <Exp>                      & operações aritméticas   \\
      <Exp>     & <Exp> "<" <Exp>                      & operações relacionais   \\
      <Exp>     & "if" <Exp> "then" <Exp> "else" <Exp> & expressão condicional   \\
      <Exp>     & "id" "(" <Exps> ")"                  & chamada de função       \\
      <Exp>     & "let" "id" "=" <Exp> "in" <Exp>      & expressão de declaração \\[.9em]
      <Exps>    & <Exp>                                & lista de argumentos     \\
      <Exps>    & <Exp> "," <Exps>                     &                         \\
    \end{tabular}
    \end{mdframed}
  \end{synshorts}
  \caption{Linguagem \lang{}}
  \label{grm:trivial}
\end{gramatica}

Esta gramática é claramente \textbf{ambígua}, como ilustra as figuras
\ref{fig:ast.op}, \ref{fig:ast.if}, e \ref{fig:ast.let}. As
ambiguidades podem são resolvidas observando
\begin{itemize}
  \item uma definição de \textbf{precedência} relativa e
  \textbf{associatividade} para dos operadores, indicadas pela
  tabela~\ref{tab:prec}, em ordem decrescente de precedência, e
  
  \item que as expressões condicionais e de declaração de variável se
  estendem ao máximo para a direita.
\end{itemize}

\begin{figure}
  \begin{center}
    \tikz[parser tree] \Tree
    [.E
      [.E [.E 2 ] + [.E 3 ] ]
      <
      [.E 4 ] ];
    \hfil
    \tikz[parser tree] \Tree
    [.E
      [.E 2 ]
      +
      [.E [.E 3 ] < [.E 4 ] ] ];
  \end{center}
  \caption{Árvores sintáticas para a cadeia \texttt{2 + 3 < 4}}
  \label{fig:ast.op}
\end{figure}

\begin{figure}
  \begin{center}
    \tikz[parser tree] \Tree
    [.E if [.E x ] then [.E 2 ] else [.E [.E 3 ] + [.E 4 ] ] ];
    \hfil
    \tikz[parser tree] \Tree
    [.E
      [.E if [.E x ] then [.E 2 ] else [.E 3 ] ]
      +
      [.E 4 ] ];
  \end{center}
  \caption{Árvores sintáticas para a cadeia \texttt{if x then 2 else 3 + 4}}
  \label{fig:ast.if}
\end{figure}

\begin{figure}
  \begin{center}
    \tikz [parser tree] \Tree
    [.E let x = [.E 2 ] in [.E [.E x ] > [.E 3 ] ] ];
    \hfil
    \tikz [parser tree] \Tree
    [.E
      [.E let x = [.E 2 ] in [.E x ] ]
      >
      [.E 3 ] ] ;
  \end{center}
  \caption{Árvores sintáticas para a cadeia \texttt{let x = 2 in x < 3}}
  \label{fig:ast.let}
\end{figure}


\begin{table}[!htb]
  \centering
  \begin{tabular}{|l|l|}\hline
    \textbf{operadores} & \textbf{associatividade} \\\hline
    \texttt{+}          & esquerda                 \\\hline
    \texttt{<}          & -                        \\\hline
  \end{tabular}
  \caption{Precedência e associatividade dos operadores}
  \label{tab:prec}
\end{table}

Um \textbf{programa} em \lang{} é uma lista de declarações de
\textbf{funções}. As funções são \textbf{mutuamente recursivas}, e
nenhuma função pode ser declarada mais de uma vez. Cada função declara
o tipo do seu resultado e o tipo e o nome de seus parâmetros
formais. Não pode haver repetição de nome na lista de parâmetros de
uma função. Funções e variáveis têm espaços de nome separados. O corpo
de uma função é uma \textbf{expressão}.

Os \textbf{tipos} de \lang{} são \texttt{int} (inteiro) e
\texttt{bool} (booleano).

Uma \textbf{expressão} pode ser:
\begin{itemize}[noitemsep]
  \item uma constante inteira
  \item uma variável
  \item uma operação aritmética
  \item uma operação relacional
  \item uma expressão condicional
  \item uma chamada de função
  \item uma expressão com uma declaração local
\end{itemize}

Comparações são definidas para inteiros e para booleanos (sendo
\emph{falso} considerado menor que \emph{verdadeiro}). Mas operações
aritméticas são definidas apenas para inteiros.

Um programa deve conter uma função chamada \texttt{main}, com um
parâmetro inteiro e que retorna um inteiro. Um programa é executado
pela chamada desta função com um argumento, que deve ser inteiro. O
resultado desta função é a saída do programa.

Ocorrências de \textbf{caracteres brancos} (espaço, tabulação
horizontal, nova linha) entre os símbolos léxicos são ignorados.

Os \textbf{literais inteiros} são formados por uma sequência de um ou
mais dígitos decimais. Não há literais inteiros negativos. São
exemplos de literais inteiros:
\begin{pygmented}[lang=text]
2014
872834
0
0932
\end{pygmented}

\textbf{Identificadores} são sequências de letras maiúsculas ou
minúsculas, dígitos decimais e sublinhados (\texttt{\_}), começando
com uma letra. Letras maiúsculas e minúsculas são distintas em um
identificador. Identificadores são usados para nomear entidades usadas
em um programa, como funções e variáveis. São exemplos de
identificadores:
\begin{pygmented}[lang=text]
peso
idadeAluno
alfa34
primeiro_nome
\end{pygmented}

Não são exemplos de identificadores:
\begin{pygmented}[lang=text]
__peso
idade do aluno
34rua
primeiro-nome
x'
\end{pygmented}

\textbf{Palavras-chave}, usadas em algumas construções sintáticas da
linguagem, são reservadas, isto é, não podem ser usadas como
identificadores.

Exemplo de programa em \lang{}:
\begin{pygmented}[lang=text]
int f(int a, int b) =
  let y = a + b
  in
    if a < b then
      f(sucessor(a), b) + y
    else
      y + y

int sucessor(int n) =
  n + 1

int main(int x) =
  f(x, 10)
\end{pygmented}

\section{O projeto}

\subsection{Estrutura do projeto}

O projeto será desenvolvido na linguagem OCaml usando a ferramenta
dune para automatização da compilação. O projeto usa algumas
ferramentas e bibliotecas externas, entre as quais citamos:
\begin{description}
  \item[ocamllex] Ocamllex é um gerador de analisador léxico. Ele
  produz analisadores léxicos (em OCaml) a partir de conjuntos de
  expressões regulares com ações semânticas associadas. É distribuído
  junto com o compilador de OCaml.
  
  \item[menhir] Menhir é um gerador de analisador sintático. Ele
  transforma especificações gramaticais de alto nível, decoradas com
  ações semânticas expressas na linguagem de programação OCaml, em
  analisadores sintáticos, também expressos em OCaml.

  \item[dune] Dune é um sistema de compilação para OCaml.

  \item[ppx_import] É uma extensão de sintaxe que permite extrair
  tipos ou assinaturas de outros arquivos de interface compilados

  \item[ppx_deriving] É uma extensão de sintaxe que facilita geração
  de código baseada em tipos, em OCaml

  \item[ppx_expect] É uma extensão de sintaxe para escrita de testes
  inline de expectataiva, em OCaml

  \item[camomile] Camomile é uma biblioteca unicode para OCaml.  
\end{description}
  
O código é organizado segundo a estrutura de diretórios mostrada na
figura~\ref{fig:dir}.
\begin{figure}[H]
  \begin{tcolorbox}
    \dirtree{%
      .1 \lang{}.
      .2 src\DTcomment{código fonte do projeto}.
      .3 lib\DTcomment{biblioteca com as principais definições usadas para montar o compilador}.
      .3 driver\DTcomment{driver do compilador}.
      .2 doc\DTcomment{Documentação de \lang{}}.
    }
  \end{tcolorbox}
  \caption{Estrutura de diretórios do projeto do compilador}
  \label{fig:dir}
\end{figure}
Existem outros diretórios gerados automaticamente que não são
relevantes nesta discussão e por isto não foram mencionados. Caso
algum ambiente de desenvolvimento integrado (IDE) seja usado,
provavelmente haverá alguns arquivos e diretórios específicos do
ambiente e que também não são mencionados.

Os arquivos \texttt{src/lib/dune} e \texttt{src/driver/dune} contém as
especificações do projeto esperadas pelo dune. Neles são indicadas
informações como nome do projeto, dependências externas e flags
necessários para a compilação da biblioteca e da aplicação.

\subsection{Módulos importantes}

Alguns módulos importantes no projeto são mencionados a seguir. Alguns
já estão prontos, e outros deverão ser criados ou modificados pelo
participante nas atividades do curso.

\begin{itemize}
  \item O módulo \pyginline|Absyn| contém os tipos que representam as
  árvores sintáticas abstratas para as construções da linguagem fonte.

  \item O módulo \pyginline|Absyntree| contém funções para converter
  árvores de sintaxe abstratas para árvores genéricas de strings,
  úteis para visualização das árvores sintáticas durante o
  desenvolvimento do compilador.

  \item O módulo \pyginline|Lexer| contém as declarações relacionadas
  com o analisador léxico do compilador.

  O analisador léxico (módulo \pyginline|Lexer|) é gerado
  automaticamente pelo \texttt{ocamllex}. A especificação léxica é
  feita no arquivo \texttt{src/lib/lexer.mll} usando expressões
  regulares.

  \item O módulo \pyginline|Parser| contém as declarações relacioandas
  com o analisador sintático do compilador.

  O analisador sintático (módulo \pyginline|Parser|) é gerado
  automaticamente pelo \texttt{menhir}. A especificação sintática é
  feita no arquivo \texttt{src/lib/parser.mly} usando uma gramática
  livre de contexto.
  
  \item O módulo \pyginline|Semantic| contém declarações usadas na
  análise semântica e geração de código do compilador.

  \item O módulo \pyginline|Symbol| contém declarações que implementam
  um tipo usado para representar nomes de identificadores de forma
  eficiente, e será discutido posteriormente.

  \item O módulo \pyginline|Env| contém declarações para a manipulação
  dos ambientes de compilação (às vezes também chamados de
  contexto). Estes ambientes são representados usando tabelas de
  símbolos.

  \item O módulo \pyginline|Types| contém declarações para a
  representação interna dos tipos da linguagem fonte.

  \item O módulo \pyginline|Error| contém declarações usadas para
  reportar errors detectados pelo compilador durante a compilação.

  \item O módulo \pyginline|Location| contém declarações usadas para
  representação de localizações de erros no código fonte, importantes
  quando os erros forem reportados.

  \item O módulo \pyginline|Driver| é formado por declarações,
  incluindo a função \pyginline|main|, ponto de entrada para a
  execução do compilador.
\end{itemize}

\subsection{Implementando uma nova construçao de \lang{}}

Ao acrescentar uma nova construção na implementação da linguagem,
procure seguir os seguintes passos:
\begin{itemize}
  \item Se necessário defina um novo construtor de dados no tipo
  \pyginline|Types.t| para representar algum tipo da linguagem \lang{}
  que ainda não faça parte do projeto. \textit{[Análise semântica]}

  \item Se necessário acrecente ao ambiente inicial (no módulo
  \pyginline|Env|) a representação de quaisquer novos tipos, variáveis
  ou funções que façam parte da biblioteca padrão de
  \lang{}. \textit{[Análise semântica]}

  \item Defina os novos construtores de dados necessárias para
  representar a árvore abstrata para a construção no tipo
  \pyginline|Absyn.t|.
  \begin{itemize}[noitemsep]
    \item definir os campos necessários para as sub-árvores da árvore
    abstrata, se houver
    \item extender as funções do \pyginline|Absyntree| para permitir a
    conversão para uma árvore de strings, útil para visualização
    gráfica da árvore abstrata.
  \end{itemize}
  \textit{[Análise sintática]}

  \item Extenda a função de análise semântica
  \pyginline|Semantic.semantic| para verificar a nova
  construção. \textit{[Análise semântica]}

  \item Declare quaisquer novos símbolos terminais e não-terminais na
  gramática livre de contexto da linguagem que se fizerem necessários
  para as especificações léxica e sintática da
  construção. \textit{[Análises léxica e sintática]}

  \item Acrescente as regras de produção para a construção na
  gramática livre de contexto da linguagem, tomando o cuidado de
  escrever ações semânticas adequadas para a construção da árvore
  abstrata correspondente. Se necessário use declarações de
  precedência de operadores. \textit{[Análise sintática]}

  \item Se necessário acrescente regras léxicas que permitam
  reconhecer os novos símbolos terminais da especificação léxica da
  linguagem. \textit{[Análise léxica]}
\end{itemize}

\section{Mensagens de erro}

O projeto contém algumas funções para \textbf{reportar erros}
encontrados durante a compilação. Estas funções fazem parte do módulo
\pyginline|Error| e serão comentadas a seguir.

Em todo compilador é desejável que os erros encontrados sejam
reportados com uma indicação da \textbf{localização} do erro,
acompanhada por uma \textbf{mensagem} explicativa do problema
ocorrido. Para tanto torna-se necessário guardar a informação da
localização em que cada frase do programa (o que corresponde a cada nó
da árvore abstrata construída para representar o programa) foi
encontrada. O módulo \pyginline|Location| contém algumas definições
relacionadas com estas localizações.

Neste projeto as localizações no código fonte são representadas pelo
tipo \pyginline|Location.t|, que leva em consideração as posições no
código fonte onde a frase começou e terminou.

Cada uma destas posições é do tipo \pyginline|Lexing.position|. O
módulo \pyginline|Lexing| faz parte da biblioteca padrão do OCaml e
será extensivamente usado nas implementações dos analisadores léxico e
sintático. O tipo \pyginline|Lexing.position| contém as seguintes
informações:
\begin{itemize}[noitemsep]
  \item a indicação da unidade (arquivo fonte) sendo compilada,
  \item o número da linha, e
  \item o número da coluna
\end{itemize}

A \textbf{função} \pyginline|Error.error|, e outras funções similares
encontradas no módulo \pyginline|Error|, devem ser usadas para emissão
de mensagens de erro. Esta função recebe como argumentos a localização
do erro, a mensagem de formatação de diagnóstico, e possivelmente
argumentos complementares de acordo com a mensagem de formatação.


\section{Símbolos}

Linguagens de programação usam \textbf{identificadores} para nomear
entidades da linguagem, como tipos, variáveis, funções, classes,
módulos, etc.

\textbf{Símbolos léxicos} (também chamados de \textbf{símbolos
  terminais} ou \emph{\textbf{tokens}}) que são classificados como
identificadores têm um valor semântico (atributo) que é o nome do
identificador. A princípio o valor semântico de um identificador pode
ser representado por uma cadeia de caracteres (tipo \pyginline|string|
do OCaml). Porém o tipo \pyginline|string| tem algumas incoveniências
para o compilador:
\begin{itemize}
  \item Geralmente o mesmo identificador ocorre várias vezes em um
  programa. Se cada ocorrência for representada por uma string (ou
  seja, por uma sequência de caracteres), o uso de memória poderá ser
  grande.

  \item Normalmente existem dois tipos de ocorrência de
  identificadores em um programa:
  \begin{itemize}
    \item uma declaração do identificador, e
    \item um ou mais usos do identificador já declarado.
  \end{itemize}
  Durante a compilação cada ocorrência de uso de um identificador deve
  ser associada com uma ocorrência de declaração. Para tanto os
  identificadores devem ser comparados para determinar se são iguais
  (isto é, se tem o mesmo nome). O uso de strings é ineficiente, pois
  pode ser necessário comparar todos os caracteres das strings para
  determinar se elas são iguais ou não.
\end{itemize}

Por estas razões o compilador utiliza o tipo \pyginline|Symbol.symbol|
para representar os nomes dos identificadores. Basicamente usa-se uma
tabela \emph{hash} onde os identificadores são colocados à medida que
eles são encontrados. Sempre que o analisador léxico encontrar um
identificador, deve-se verificar se o seu nome já está na tabela. Em
caso afirmativo, usa-se o símbolo correspondente que já se encontra na
tabela. Caso contrário cria-se um novo símbolo, que é adicionado à
tabela associado ao nome encontrado, e este novo símbolo é usado pelo
analisador léxico como valor semântico do \emph{token}.

Na implementação de \pyginline|Symbol.symbol| associa-se a cada novo
símbolo um número inteiro diferente. A comparação de igualdade de
símbolos se resume a uma comparação (muito eficiente) de inteiros, já
que o mesmo identificador estará sempre sendo representado pelo mesmo
símbolo (associado portanto ao mesmo número inteiro).

A função \pyginline|Symbol.symbol| cria um símbolo a partir de uma
string, e a função \pyginline|Symbol.name| obtém o nome de um símbolo
dado.


\section{O  analisador léxico}

O módulo \pyginline|Lexer| contém as declarações que implementam o
analisador léxico do compilador. Este módulo será gerado
automaticamente pela ferramenta \texttt{ocamllex}.

A especificação da estrutura léxica da linguagem fonte é feita no
arquivo \texttt{src/lib/lexer.mll} usando \textbf{expressões
  regulares}. Consulte a
\href{https://ocaml.org/releases/4.12/htmlman/lexyacc.html}{documentação
  do \texttt{ocamllex}} para entender como fazer a especificação
léxica.

Os analisadores léxico e sintático vão se comunicar durante a
compilação, pois os tokens obtidos pelo analisador léxico serão
consumidos pelo analisador sintático. Ou seja, os tokens são os
símbolos terminais da gramática usada pelo gerador de analisador
sintático. Para manter a consistência dos analisadores léxico e
sintático \emph{os símbolos terminais (tokens) são declarados na
  gramática livre de contexto} do \texttt{menhir}, no arquivo
\texttt{src/lib/parser.mly}. Consulte a documentação do
\texttt{menhir} para entender como escrever a gramática livre de
contexto.



\end{document}

% Local Variables:
% ispell-local-dictionary: "brazilian"
% End:
